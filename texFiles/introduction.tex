\section*{1. Introduction}
\addcontentsline{toc}{section}{1. Introduction}

\subsection*{Product Vision}
\addcontentsline{toc}{subsection}{Product Vision}
The target audience of our app are SETU students who have access to Moodle specifically, who are looking for a space where all their resources and different applications are easily accesible.
\hfill \break

The SETUP app (also known as South East Techonological University Personalised) is an application that acts as a hotspot between multiple SETU services, allowing you to access anything from exam papers to lecture contacts to Outlook to Moodle, all within the one space that include important features such as discussion groups, calendars, class timetables, and a map feature. 
\hfill \break

Unlike Moodle, which already has access to a lot of these, it is formatted in a way that makes it easier to navigate and makes all essential services centralised while also letting departments and services, such as the Computer and Maths Learning Centre, to get in touch with students directly and notify them of events and such.
\hfill \break

In our own experience, having resources split across multiple platforms is extremely hard to work with, but with SETUP, both lecturers and students can stay in frequent contact and on top of work and schedules.
\hfill \break

Our Product, SETUP, gives SETU students a completely integrated, secure and student-focused experience that brings campus services, academic information and updates that happen in real time into one place.

\subsection*{Similar App A [PLACEHOLDER]}
\addcontentsline{toc}{subsection}{Similar App A [PLACEHOLDER]}
move to your branch for this, remember to merge branches when complete

\subsection*{Similar App B - Studo}
\addcontentsline{toc}{subsection}{Similar App B [PLACEHOLDER]}

The main thing Studo offers is an app which combines both academic and campus services into one place.
For example, SETU Timetables for students and Moodle are two seperate platforms, but Studo combines these together and lets you access them from the same application.
They have partnerships and supply software to over 60 different universities across Germany and Austria.
Their mobile app has over 500,000 downloads, not including desktop downloads.

\hfill \break
Good things about Studo:
\begin{itemize}
  \item Combines lots of daily university/on-campus services (email, calendar, grades, timetables)
  \item Real-Time polls and push alerts
  \item Easy to integrate existing records and student information
  \item Easily acessable course and grade overview with key dates for assignments and uploads
  \item Both synced and accesible from multiple devices
  \item University news feed and up-to-date food menus
\end{itemize}

Bad things about Studo:
\begin{itemize}
  \item The overall application is extremely basic. Many of the features are extremely limited in what they can do.
  \item Many tools and featurs are admin controlled, not easy for students to customise what they see, etc
  \item Not accesible for students, looking at reviews it looks like students struggle connecting their student accounts to the application
  \item The calander is limited on what you can add/info you add to it
  \item Chat feature between students is heavily moderated and deletes things that don't go against the platforms rules
  \item 2FA and login features are unreliably and often don't work, affecting time pressured situations
\end{itemize}

\subsection*{Features}
\addcontentsline{toc}{subsection}{Features}
etc. etc.