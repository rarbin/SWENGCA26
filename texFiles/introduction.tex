\section*{1. Introduction}
\addcontentsline{toc}{section}{1. Introduction}

\subsection*{Product Vision}
\addcontentsline{toc}{subsection}{Product Vision}
As students of SETU, we aim to develop a dedicated, accessible student services product, that unifies the apps and utilities common to daily student life. The ideal SETU Personalised (SETUP) application integrates both existing and new educational and social features alike, giving staff \& students more control over their educational experience, in real time, in one place. \\

To gain a better understanding of what we want SETUP to be, we investigated three existing products; Maynooth University App, xTiles \& Studo. While each differs in scope and functionality, all serve as a reference point for different features SETUP implements. \\ 

\begin{minipage}{0.65\textwidth}
\subsection*{A1. Maynooth University App}

\addcontentsline{toc}{subsection}{A1. Maynooth University App}

    Staff and students enrolled in Maynooth University may avail of the cross-platform MU App for its variety of educational and social features.\cite{MUA} As a service unique to the university, app features can be easily tailored to the needs of its 1k+ userbase, such as:\\

    \begin{itemize} 
        \item Campus Map Navigation
        \item MU Eduroam Setup
        \item University News \& Events
        \item MU Clubs \& Socs
        \item Commuting Info
        \item MU Eats
    \end{itemize}

    This comprehensive design distinguishes itself from the more generalised alternatives on the market. By mitigating the reliance on third-party providers, MU has more flexibility in adapting to the needs of its users. \\

    As a student life \& learning centred platform supported by the college, it is also free of costly subscription charges in order to have full functionality of the app.\\

\end{minipage}%
\hfill
\begin{minipage}{0.28\textwidth}
    \centering
    \includegraphics[width=\linewidth]{image/MUA1.png}
    \vspace{0.5em}
    \footnotesize{\textbf{Fig 1.} MU App Hompeage}
\end{minipage}

Released in 2022, some features still feel relatively underdeveloped.\cite{GPM} Products that take inspiration from MU's design could benefit from iterating upon the educational tools provided. The app currently lacks Moodle integration, despite Maynooth's usage of the platform.\cite{MMO}\\

For instance, the student service 'Studo' \cite{studo}. integrates existing features from several apps into the one front-end interface. While this increases the reliance on third-parties, it's important to be realistic about development cost, time and ability. Often, it can be more redundant than useful to try reinvent the existing available tools,such as Moodle's learning environment, especially as a smaller team.\\

SETU stands to benefit in using the Maynooth University App as a reference point for similar development, given the comparable scale, resources and services ideal for both. 

\subsection*{A2. xTiles}
\addcontentsline{toc}{subsection}{A2. xTiles}
xTiles is a workspace and visual productivity platform that is aimed towards daily planning, personal tasks, note-taking and project management.\cite{xtiles} Compared to more document formatted alternatives, xTiles allows students and creative professionals to visualise their workflow with a variety of templates and widgets.\\

The feature of drag-and-drop like moving texts, links and images makes it easy to rearrange
your workspace. It is user-friendly and suitable for creating schedules and weekly study schedules. As a target demographic, students are provided special offers on paid plans.

\vspace*{-\baselineskip}
\begin{center}
\begin{minipage}{\textwidth}
    \centering
    \includegraphics[width=450px]{image/xtiles.png}
    \footnotesize{\textbf{Fig 2.} xTiles Interface}
\end{minipage}
\end{center}
\hfill \break
When developing the SETUP student service it's worth considering the following aspects of xTile's design approach that work well:\\
\begin{itemize}
    \vspace{-0.5\baselineskip}
\item \textbf{Flexible} visual workspace where you can customize to your own needs.\\ (Enables the user to add texts, images, links, etc.)
\item \textbf{Real-time} multi-user support for collaboration.\\ (Teams can work on a board together.)
\item \textbf{Utilities} for making plans, brainstorms and organising.
\item \textbf{Cross-platform}, accessible through multiple devices.
\item \textbf{Integration} with other applications e.g. Google Calendar.\\ (Allows events and deadlines to be synced across project.)
\item  \textbf{Templates} provided, giving users the option of customisation without the expectation\\ of starting from scratch.
\item Visually appealing and \textbf{intuitive UI} improves upon \textbf{accessibility} and engagement \\for ADHD, low-vision, etc.
\end{itemize}
Alongside the reported limitations\cite{xtrev} of the platform:
\begin{itemize}
\item  Lacking in \textbf{offline support}, making it unreliable for those without consistent internet.
\item The large variety in tools can be offputting to new users without \textbf{clear instructions}. 
\item \textbf{Limited free plan} often pushes users into paid alternatives to access what some may expect to be basic features.
\end{itemize}

It's difficult for any project to predict how any given feature will be received or what challenges it may bring to development. For this reason, being receptive to user feedback at all stages of development can be hugely beneficial in the long-term. \\

\begin{minipage}{0.65\textwidth}
\subsection*{A3. Studo}
\addcontentsline{toc}{subsection}{A3. Studo}
Studo combines both academic and campus services into one place. The company partners with \& supplies software across 60+ universities, with over 500k downloads on the mobile app version of its Studo student service.\cite{studo}\\ 

As opposed to developing the platforms independently, the Studo app focuses on combining existing, but separate, platforms into one front-end application. e.g. SETU Timetable, student grades \& Moodle access. \\

Notable features include:\\
\begin{itemize}
    \vspace{-0.5\baselineskip}
  \item \textbf{All-in-one} interface for daily university/on-campus services \\(email, calendar, grades, timetables)
  \item \textbf{Real-Time} polls and push alerts
  \item \textbf{Easy integration} for existing records and student information
  \item \textbf{Accessible} course and grade overview. \\ (highlights dates for assignments and uploads)
  \item \textbf{Synced} cross-platform from multiple devices
  \item University news feeds, vouchers, lunch menus \\
\end{itemize}

\end{minipage}%
\hfill
\begin{minipage}{0.28\textwidth}
    \centering
    \includegraphics[width=\linewidth]{image/studo.png}
    \vspace{0.5em}
    \footnotesize{\textbf{Fig 3.} Studo Menu}
\end{minipage}

However, it's often criticised for: 
\begin{itemize}
  \item \textbf{Limited} or \textbf{redundant} features, many of which are heavily admin restricted.
  \item \textbf{Lack of customisation} options for students
  \item Chat feature between students is heavily moderated and \textbf{overly censorious}. 
  \item 2FA and login features are unreliable, reported difficulty in students connecting accounts among other \textbf{integration issues}.
\end{itemize}

\subsection*{Features}
\addcontentsline{toc}{subsection}{Features}
etc. etc.