\section*{1. Introduction}
\addcontentsline{toc}{section}{1. Introduction}

\subsection*{Product Vision}
\addcontentsline{toc}{subsection}{Product Vision}

The target audience of our app are SETU students who have access to Moodle specifically, who are looking for a space where all their resources and different applications are easily accesible.
\hfill \break

The SETUP app (also known as South East Techonological University Personalised) is an application that acts as a hotspot between multiple SETU services, allowing you to access anything from exam papers to lecture contacts to Outlook to Moodle, all within the one space that include important features such as discussion groups, calendars, class timetables, and a map feature. 
\hfill \break

Unlike Moodle, which already has access to a lot of these, it is formatted in a way that makes it easier to navigate and makes all essential services centralised while also letting departments and services, such as the Computer and Maths Learning Centre, to get in touch with students directly and notify them of events and such.
\hfill \break

In our own experience, having resources split across multiple platforms is extremely hard to work with, but with SETUP, both lecturers and students can stay in frequent contact and on top of work and schedules.
\hfill \break

Our Product, SETUP, gives SETU students a completely integrated, secure and student-focused experience that brings campus services, academic information and updates that happen in real time into one place.\\

\begin{minipage}{0.65\textwidth}
\subsection*{A1. Maynooth University App}

\addcontentsline{toc}{subsection}{A1. Maynooth University App}

    Staff and students enrolled in Maynooth University may avail of the cross-platform MU App for its variety of educational and social features.\cite{MUA} As a service unique to the university, app features can be easily tailored to the needs of its 1k+ userbase, such as:\\

    \begin{itemize} 
        \item Campus Map Navigation
        \item MU Eduroam Setup
        \item University News \& Events
        \item MU Clubs \& Socs
        \item Commuting Info
        \item MU Eats
    \end{itemize}

    This comprehensive design distinguishes itself from the more generalised alternatives on the market. By mitigating the reliance on third-party providers, MU has more flexibility in adapting to the needs of its users. \\

    As a student life \& learning centred platform supported by the college, it is also free of costly subscription charges in order to have full functionality of the app.\\

\end{minipage}%
\hfill
\begin{minipage}{0.28\textwidth}
    \centering
    \fbox{\includegraphics[width=\linewidth]{image/MUA1.png}}
    \vspace{0.5em}

\end{minipage}

Released in 2022, some features still feel relatively underdeveloped. \cite{GPM} Products that take inspiration from MU's design could benefit from iterating upon the educational tools provided. The app currently lacks Moodle integration, despite Maynooth's usage of the platform. \cite{MMO}\\

For instance, the student service 'Studo' \cite{studo}. integrates existing features from several apps into the one front-end interface. While this increases the reliance on third-parties, it's important to be realistic about development cost, time and ability. Often, it can be more redundant than useful to try reinvent the existing available tools,such as Moodle's learning environment, especially as a smaller team.\\

SETU stands to benefit in using the Maynooth University App as a reference point for similar development, given the comparable scale, resources and services ideal for both. 

\subsection*{A2. XTiles}
\addcontentsline{toc}{subsection}{A2. XTiles}
XTiles is a workspace and visual productivity platform that is aimed towards daily planning, personal tasks, note-taking and project management. 
It is more of a visual board rather than a document and because of this it is focused toward students and creative professionals who prefer to 
see things visually organised. XTiles feature of drag-and-drop like moving texts, links and images makes it easy to arrange around
your workspace. It is user friendly and suitable for customers to create schedules and their weekly study plans. 

\hfill \break
Good things about xTiles:
\begin{itemize}
\item Flexible visual workspace where you can customize to your own needs. (Enables the user to add texts, images, links, etc.)
\item It supports real-time multi-user collaboration (Teams can work on a board together).
\item Great to make plans, brainstorms and organizing.
\item It is cross-platform, so you can access it through multiple devices.
\item It’s integrated with other applications like Google Calendar. Allows to sync with events or deadlines.
\item  With the templates they provide, they are visually pleasing and excellent for those with ADHD.
\item Special offers for students.
\end{itemize}
\hfill \break
Bad things about xTiles:
\begin{itemize}
\item  There isn’t a lot of offline activities to do, it is rather more internet accessible.
\item It can be overwhelming at first because there is numerous of tools to go through and learn.
\item The free plan is fairly limited and must pay for unlimited workspaces. 
\end{itemize}

\subsection*{Features}
\addcontentsline{toc}{subsection}{Features}
etc. etc.